

\section{Introduction}

Everyone knows the weather, but few people have actually analyzed the data we publicly have access to. We decided to look at this data and see what could actually be inferred from it. We set out to do 3 investigations:
\begin{enumerate}
    \item Climate Trends
    \item Birthday Temperature Trends
    \item Solar Activity Period Estimation
\end{enumerate}

We are using the Swedish Meteorological and Hydrological Institute (SMHI), since that is the country we live in.

\subsection{Climate Trends}
We study how Sweden’s temperature has changed from 1850 to 2024 by analyzing trends in maximum, minimum, and mean temperatures for both the Lund station and the country as a whole to quantify the general climate trend.

\subsection{Birthday Temperature Trends}


{\Huge \textcolor{red}{We are born \\ and \\ We die}}\\\textcolor{red}{/Homer Simpson (perchance)}\\

We were curious to see how the temperature has historically changed over our birthdays [todo].

\subsection{Solar Activity Period Estimation}
The solar activity is usually measured by looking at the number of sunspots on the sun. Instead of this we wanted to see if this increase in activity, and thus luminosity would impact the temperature in Sweden and if this period could be inferred.

% Purpose: Briefly explain why studying temperature trends over long periods is important (e.g., climate change indicators, regional variation, etc.).

% Scope: Describe what the project investigates:

% Long-term mean temperature evolution in Sweden

% Temperature trends on specific calendar dates (e.g., birthdays)

% Possible correlations with solar events

% Data Source: SMHI open climate dataset (1850–2024).